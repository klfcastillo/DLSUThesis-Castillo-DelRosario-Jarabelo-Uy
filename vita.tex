\chapter{Vita}

% Change the descriptions accordingly

{
\vfill

\includegraphics[width=0.2\columnwidth]{vita_photo1}
\documentAuthor{firstname1} \ \documentAuthor{surname1} \  is currently studying B. S. Computer Engineering in De La Salle University, Malate, Manila, Philippines. He is a senior member of the Electrical Team of the DLSU Eco Car Team, which is composed of students who apply their knowledge in their respective fields to be able to create innovative energy efficient vehicles and to spread awareness on green technology. He has an experience in various programming languages such as C, Java, and Android and has developed projects that involved the use of the PIC microcontroller, which is his major research interest.

\includegraphics[width=0.2\columnwidth]{vita_photo2}
\documentAuthor{firstname2} \ \documentAuthor{surname2} \ is a 4th year engineering student and currently 
taking up Bachelor of Science in Computer Engineering in De La Salle University - Manila, Malate, Manila, Philippines. He is knowledgeable in programming languages (such as C++, Java, HTML), PIC programming, PCB fabrication and circuit analysis and construction.

\includegraphics[width=0.2\columnwidth]{vita_photo3}
\documentAuthor{firstname3} \ \documentAuthor{surname3} \ is a 4th year Computer Engineering student is De La Salle University Manila - Manila, Malate,Manila, Philippines. Has knowledge in C++, JAVA, MS OFFICE, microcontroller programming and PCB fabrication. A member of the organization ACCESS holding the position of internal Vice President which is an organization for Computer Engineering students.

\includegraphics[width=0.2\columnwidth]{vita_photo4}
\documentAuthor{firstname4} \ \documentAuthor{surname4} \ is a 4th year engineering student and currently  taking up Bachelor of Science in Computer Engineering at De La Salle University - Manila, Philippines. He is knowledgeable in database systems, programming languages such as C, CSS, Java, HTML, MySQL,  PHP and Verilog, and has developed several electronic circuits that utilizes the PIC microcontroller. His research interest includes automation, environment friendly technologies and radio frequency identification  technologies.


\vfill
}