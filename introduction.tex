\section{Background of the Study}

	Commonly used today RFID originated as use as early as used in war and espionage in the earlier age. [1]Several countries, namely the British, used radars to detect different approaching planes from a faraway. The problem is they had no way of detecting whose plane it belongs they only find out when it lands as they returned to base. From a signal that couldn’t be read now it is part of everyday life. They installed a transmitter in a plane, this receive a signal from the radar stations installed. This was used to identify if the plane was there if a signal was received by them. This technology was thought to be only available to big companies. . This technology can be compared to out RFID we have today. [3]Wal-Mart was one of the 1st companies that adapted this kind of technology. They used it as to improving in monitoring there inventory, but in a while this didn’t meet the expectation they expected to improve their inventory. RFID started to emerge commercially. Now the RFID can be found and used by a lot of people ,when it became commercially available this technology is common, but a very useful one indeed. 
[2]This increases the productivity of our everyday activities. RFID is used if not all by a lot.[2] This is a convenient way of avoiding traffic, theft, faster toll movement, etc. The uses of these are not limited to car readings in expressways these are also used for in parking; getting in buildings; campuses; airports; and many more. RFID is a term taken from short range communication.[3] The RFID is normally composed and based on the tag, a reader, and a database. A reader can be setup on a certain place. The tag has a small chip/ or fixed radio antenna. A wave is transmitted between the tag and reader. A signal is sent and the transponder reacts, and then reflects the signal back on its broadcast signal.

\section{Prior Studies}
`
Put here a \index{summary}summary of your literature review.  Preferably, a table showing the summary would be helpful. 

RFID technology is already being implemented worldwide. There are numerous studies on RFID that involves either using active or passive RFID. 
\section{Problem Statement}

Attendance checking has always been a difficult and hassle for professors using the pen and paper method or the so called “traditional way”. Roll calling or calling each of the students in the class can be tedious especially when facing with big numbers of students in a lecture class who may not all be focused and may thus miss the opportunity to be called by the professor. Furthermore, the traditional method may eat up more than needed time for lecture and such.  
	In this research, the researchers are using RFIDs as an aid for checking attendances and the regulation of students from getting in and out of the class room.  The research should aim to get the each student’s data (ID number) by scanning their IDs efficiently so that it can help and assist the professor in getting attendances and to prevent any discrepancies using the traditional way of getting attendances.  The procedure or method of scanning IDs should be enhanced for efficiency but in the same way should be less hassle so that the attendance should be not time consuming.  
	Attendance by the students are going to be more efficient and less hassle because using RFID tags, student data will be acquired by passing through the door since RFID tags are capable to be sensed by reader in a given distance.  Thus it can avoid and lessen the difficulties faced by the professors using the traditional method of getting attendances.  The system also features the time logging for the professor to determine the time when students arrived in the lecture room which can be used for marking absence or tardiness of some students.  

\section{Objectives}
\subsection{General Objective(s)}
To \ldots;

\subsection{Specific Objectives}

\begin{enumerate}
	\item To  \ldots;
	
	\item To  \ldots;
	
	\item To  \ldots;
	
	\item To  \ldots;
	
	\item To  \ldots;
\end{enumerate}



\section{Significance of the Study}

As RFID can provide a reliable yet inconspicuous identification for entities, its use in the world of today has been limitless. From grocery items to IT industry materials, RFID has been used to aid in the logistics of items and products of companies to provide more efficient services to consumers. This technology has also seen implementation in tracking participants of races and marathons as well as people in amusement parks and attendees in concerts (Bluemner, 2014).
Attendance checking has been a hassle for teachers using the traditional method of pen and paper. Calling each and every one of the students on the roll can be tiresome especially when dealing with large numbers of students in a lecture class who may not all be attentive and may thus miss the chance to make known their presence to the professor. Additionally, this method may consume more than necessary time – time that may be instead spent on imparting knowledge to students.
As RFID tags do not require swiping or tapping on a sensor, the attendance of students may be easily taken while they pass through the door as such tags can be sensed by readers from a certain distance. This therefore avoids the problems encountered by instructors with regards to the pen and ink method: no class time consumed for attendance checking, all students present accounted for in the attendance, and finally, the professor would conserve their energy by no longer reciting all the names in the list. The instructor can then dedicate more time to lecture and activities. The system can also be used to log the time that the student entered the classroom and therefore the professor would have a clear view of who among the students arrived on time in the class. As attendance is now digital, the transfer of data would be lesser prone to clerical error that are introduced by the typist who manually inputs the data into the computer since they would no longer be involved in the process.
The implementation of RFID to the checking of attendance would impart to the students the value of punctuality and time management. When students see that such technology is being utilized for such a petty action, they would realize that attendance is an important part of being a student. This would lead them to avoid skipping classes and to give appropriate attention to studies inside the classroom. 

\section{Assumptions, Scope and Delimitations}

Bulletize your scope in one group, and then bulletize the delimitations in another.  Bulletize your assumptions as well.


\section{Description and Methodology}

\blindtext


\ifFinished
\else

\section{Estimated Work Schedule and Budget}

Gantt chart or similar is to be part of this section.

\blindtext

\section{Publication Plan}
\blindtext

\fi


\section{Overview}

Provide here a brief summary and what the reader should expect from each succeeding chapter.  Show how each chapter are connected with each other.

